As automated driving assistance and fully automated car systems get more relevant and prevalent, intelligent object classification in the surroundings of roadways becomes increasingly important. Therefore road detection, which means that we visually classify the area in front of a car as drivable or not-drivable, is one of the basics task that has to be solved. In recent years multiple methods were proposed with different focus on precision, time efficiency or flexibility. This final report about the practicum I did at the Heidelberg Collaboratory for Image Processing describes such a method. Using multiple structural features two classifiers are trained from training data. These trained models are used to segment frames of videos. Training and predicting is done on videos from a camera that is fixed at the windscreen of a car. The general difficulty with this approach is that roadways are very diverse and the conditions under which the video is taken are changing constantly. A road can be a highway with clearly defined roadside or a gravel path without markings and varying shapes and the video can be recorded under a strong shadow casting sun or through rain. Therefore a method has to be more flexible successful then with for example simple supervised topological structuring.